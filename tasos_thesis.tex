\documentclass[12pt]{rockefeller}
\usepackage[english]{babel}
\usepackage[pdftex]{graphicx}\graphicspath{{imgs/}}
\usepackage{helvet}
\renewcommand\familydefault{\sfdefault} 
\usepackage{fancyhdr}
\usepackage[utf8x]{inputenc}
\usepackage[LGR,T1]{fontenc}
\newcommand{\textgreek}[1]{\begingroup\fontencoding{LGR}\selectfont#1\endgroup}
\usepackage{hyperref} 
\usepackage[acronym,xindy,toc]{glossaries} % nomain, if you define glossaries in a file, and you use \include{INP-00-glossary}
\makeglossaries
\usepackage[xindy]{imakeidx}
\makeindex



%\usepackage{enumerate}
%\usepackage{threeparttable}
%\usepackage{multirow}
%\usepackage[algoruled,linesnumbered,lined]{algorithm2e}
%\usepackage[font=footnotesize,caption=false]{subfig}
%\usepackage{amssymb}
%\usepackage{amsmath}
%\usepackage[pdftex,bookmarks=true]{hyperref}
%\newcommand{\subsubsubsection}[1] {\noindent {\underline {#1}}}
	\newcommand{\snp}[1] {\noindent {\underline {#1}}}

\begin{document}

\author{Tasos Gogakos}
\title{\MakeUppercase{Characterizing human transfer rnas by hydro-trnaseq and par-clip}}
\date{June 2017}

\maketitle

\begin{abstract}

The participation of transfer RNAs (tRNAs) in fundamental aspects of biology and disease necessitates an accurate, experimentally confirmed annotation of tRNA genes, and curation of precursor and mature tRNA sequences. This has been challenging, mainly because RNA secondary structure and nucleotide modifications, together with tRNA gene multiplicity, complicate sequencing and sequencing read mapping efforts. To address these issues, we developed hydro-tRNAseq, a method based on partial alkaline RNA hydrolysis that generates fragments amenable for sequencing. To identify transcribed tRNA genes, we further complemented this approach with Photoactivatable Crosslinking and Immunoprecipitation (PAR-CLIP) of SSB/La, a conserved protein involved in pre-tRNA processing. Our results show that approximately half of all predicted tRNA genes are transcribed in human cells. We also report predominant nucleotide modification sites, their order of introduction, and identify tRNA leader, trailer and intron sequences. By using complementary sequencing-based methodologies we present a human tRNA atlas, and determine expression levels of mature and processing intermediates of tRNAs in human cells.
\end{abstract}

\thispagestyle{empty}
\makecopyright

%Dedication
\chapter*{} %blank chapter, no title, not included in table of contents
%\thispagestyle{empty} %no page number
\addtocounter{page}{1} %ignore this page when counting
\vspace{1.5in} %start the dedication halfway down
\begin{flushright} %center everything
\emph{\textgreek{stous goneis}} %the wonderful words
\end{flushright}

\chapter*{Acknowledgments}

First, I would like to thank my 

\tableofcontents

\listoffigures

\listoftables



    
%\chapter*{Glossary}


\mainmatter
\pagestyle{fancy}
\fancyhf{}
\lhead{\chaptername\ \thechapter}
\rhead{\thesection}
\rfoot{\thepage}

\chapter{Sample Chapter}
\section{New section}
\newacronym[longplural={Frames per Second}]{fpsLabel}{FPS}{Frame per Second}
\newglossaryentry{computer}
{
  name=computer,
  description={is a programmable machine that receives input,
               stores and manipulates data, and provides
               output in a useful format}
}
\newacronym{lvm}{LVM}{Logical Volume Manager}
\newglossaryentry{naiive}
{
  name=na\"{\i}ve,
  description={is a French loanword (adjective, form of naïf)
               indicating having or showing a lack of experience,
               understanding or sophistication}
}

Introduction
Microprocessors are ubiquitously deployed in applications ranging from commodity devices to mission critical systems, and while malfunctions in the former may cause no \newpage harm other than inconvenience, the slightest malfunction in the latter may have catastrophic consequences. \gls{computer}, \glsreset{lvm}

this is \acrshort{lvm}
\newpage

\cite{Arimbasseri:2016ey}
\chapter{woohooo}
this is a new chapter\newpage
this is pager 2
\printglossary[title=List of Terms,toctitle=Terms and abbreviations]

\newpage
\renewcommand{\bibname}{References}
\bibliographystyle{IEEEtran}
\bibliography{/Users/tasos/work/thesis/toupoutsou/tasos_v1.bib}

\addcontentsline{toc}{chapter}{References}

\end{document}
